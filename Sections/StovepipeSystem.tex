\section{Stovepipe System}

\subsection{Form \& Causes}
Stovepipe System Anti-Pattern occur within one system. Two definitions can be found:
\begin{itemize}
\item Term describes system developed to solve a specific problem and containing data that cannot be easily shared with other systems \cite{c2com}.
\item System where no abstraction layers are used to communicate between subsystem which results in situation where everything is connected with everything \cite{Virtual}.
\end{itemize}
The first one is more general and applies to all stovepipe anti-patterns.
The second one will be further discussed in next chapters.

It happens in systems which integrates different subsystems to achieve they goal. But direct causes of this anit-pattern is no architectural vision which leads to lack of already mentioned abstraction in the system providing tight coupling between implemented subsystems \cite{SurvivalGuide}.


\subsection{Distinction}

When describing stovepipe anti-patterns the clear distinction between them should be provided. Example difference are listed in \hyperref[tab:Distinction]{figure \ref{tab:Distinction}}.

\def\arraystretch{1.6}
\begin{center}
\begin{figure}[!h]
	\begin{tabular}{|p{.48\linewidth}|p{.48\linewidth}|}
	    \hline
	    Stovepipe Enterprise & Stovepipe System \\
	    \hline
	    Multiple systems & One system \\
	    Lack of common layer between systems & Lack of common abstract layer between subsystems \\
	    Caused by: & Caused by: \\
	    \ - Lack of communication & \ - Lack of architectural vision \\
	    \ - Lack of standards & \  - Tight coupling between classes \\
	    \hline
	\end{tabular}
\caption{Distinction between Stovepipe Enterprise and Stovepipe System}
\label{tab:Distinction}
\end{figure}
\end{center}

\subsection{Symptoms \& Consequences}

\subsection{Example}

\subsection{Solution}

\subsection{Exceptions}